% Tamanhos
% \tiny
% \scriptsize
% \footnotesize
% \small 
% \normalsize
% \large 
% \Large 
% \LARGE 
% \huge
% \Huge

% Posicionamento
% \centering 
% \raggedright
% \raggedleft
% \vfill 
% \hfill 
% \vspace{Xcm}   % Colocar * caso esteja no começo de uma página. Ex: \vspace*{...}
% \hspace{Xcm}

% Estilo de página
% \thispagestyle{<<nosso>>}
% \thispagestyle{empty}
% \thispagestyle{plain}  (só número, sem cabeço)
% https://www.overleaf.com/learn/latex/Headers_and_footers

% Compilador que permite usar fonte de sistema: xelatex, lualatex
% Compilador que não permite usar fonte de sistema: latex, pdflatex

% Definindo fontes
% \setmainfont{Times New Roman}  % Todo o texto
% \newfontfamily\avenir{Avenir}  % Contexto

\begingroup\thispagestyle{empty}\vspace*{-.01\textheight}\parindent=0pt 
              \formular
              \huge 
              \textbf{A estranha força\\da canção}\\\baselineskip=.67\baselineskip 

              \medskip
              
              \LARGE
              Mário de Andrade
              
              \vspace{4cm}              

              \newfontfamily\minion{Minion Pro}
              {\selectfont\minion\small Marcos Lacerda (\textit{organização})}
              
              \vspace{0.5cm}

              {\selectfont\minion\footnotesize
              1ª edição}
                    
              \vfill
              
              \begin{wrapfigure}{r}{6.5cm}
              \vspace*{-1.22\baselineskip}
              \includegraphics[width=3.5cm]{./logoacorde.png}
              \end{wrapfigure}

              \newfontfamily\timesnewroman{Times New Roman}
              {\fontsize{30}{40}\selectfont \timesnewroman hedra}
              
              \medskip

              {\selectfont\minion\small
              São Paulo \quad\the\year}
\endgroup
\pagebreak

\begingroup 

\footnotesize\parindent0pt\parskip5pt\thispagestyle{empty} 
\vspace*{.1\textheight}\mbox{} \vfill
\baselineskip=.92\baselineskip
\thispagestyle{empty}

\textbf{Mário de Andrade} (1893--1945) publicou muito e dispersamente sobre o tema da música. Foi um músico de formação erudita que, desde jovem, em 1916, com 22 para 23 anos, foi professor na área; por toda a vida manteve acesa a atenção para os fenômenos musicais não-eruditos, que vão do popular mais elementar, como uma cantiga ancestral de roda, ao mais elaborado, como a canção gravada em discos, eventualmente orquestrada para músicos que sabiam ler partitura, vendida com sucesso no mercado e validada pelo assobio anônimo. Entre os vários pontos extremos desse quadro – o polo erudito e o popular, tradição escrita e tradição oral, instrumentos de concerto e instrumentos de batuque, campo e cidade, sagrado e profano, individual e grupal –, mundos inteiros se apresentam e simbolizam a experiência humana em seus múltiplos e contraditórios aspectos, e da mesma forma mundos inteiros se cruzam, se inseminam, se realimentam, por variados caminhos e com diferentes intensidades. 

\textbf{A estranha força da canção} Tomemos a régua da história da canção no Brasil para pensar no caso concretamente. Em seu tempo de vida, Mário primeiro ouviu incontáveis discos, primeiro gravados em sistema mecânico, que em nosso país começaram a aparecer em 1902 – foi então que se definiu a duração aproximada de 3 minutos para a canção gravada, porque era o que cabia em cada gravação –, e de 1927 em diante registrados em sistema elétrico. Quase alcançou a gravação em sistema de alta fidelidade na reprodução, que de fato só depois de 1945 se tornou acessível no mercado.

\textbf{Marcos Lacerda} é sociólogo e ensaísta. Foi Diretor de Música da Funarte, responsável pelas políticas para a música de âmbito nacional. Publicou como autor o livro \textit{Hotel Universo: a poética de Ronaldo Bastos} (2019); e como organizador os livros \textit{Música: ensaios brasileiros contemporâneos} (2016) e \textit{A canção como música de invenção} (2018). É um dos curadores da coleção Cadernos Ultramares e um dos editores da revista de crítica musical \textit{Uma canção}. 

\endgroup
\pagebreak
