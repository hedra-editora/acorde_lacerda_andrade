% Tamanhos
% \tiny
% \scriptsize
% \footnotesize
% \small 
% \normalsize
% \large 
% \Large 
% \LARGE 
% \huge
% \Huge

% Posicionamento
% \centering 
% \raggedright
% \raggedleft
% \vfill 
% \hfill 
% \vspace{Xcm}   % Colocar * caso esteja no começo de uma página. Ex: \vspace*{...}
% \hspace{Xcm}

% Estilo de página
% \thispagestyle{<<nosso>>}
% \thispagestyle{empty}
% \thispagestyle{plain}  (só número, sem cabeço)
% https://www.overleaf.com/learn/latex/Headers_and_footers

% Compilador que permite usar fonte de sistema: xelatex, lualatex
% Compilador que não permite usar fonte de sistema: latex, pdflatex

% Definindo fontes
% \setmainfont{Times New Roman}  % Todo o texto
% \newfontfamily\avenir{Avenir}  % Contexto

\begingroup\thispagestyle{empty}\vspace*{-.01\textheight}\parindent=0pt 
              \formular
              \huge 
              \textbf{A estranha força\\da canção}\\\baselineskip=.67\baselineskip 

              \medskip
              
              \LARGE
              Mário de Andrade
              
              \vspace{4cm}              

              \newfontfamily\minion{Minion Pro}
              {\selectfont\minion\small Luís Augusto Fischer (\textit{organização})}
              
              \vspace{0.5cm}

              {\selectfont\minion\footnotesize
              1ª edição}
                    
              \vfill
              
              \begin{wrapfigure}{r}{6.5cm}
              \vspace*{-1.22\baselineskip}
              \includegraphics[width=3.5cm]{./logoacorde.png}
              \end{wrapfigure}

              \newfontfamily\timesnewroman{Times New Roman}
              {\fontsize{30}{40}\selectfont \timesnewroman hedra}
              
              \medskip

              {\selectfont\minion\small
              São Paulo \quad\the\year}
\endgroup
\pagebreak

\begingroup 

\footnotesize\parindent0pt\parskip5pt\thispagestyle{empty} 
\vspace*{.1\textheight}\mbox{} \vfill
\baselineskip=.92\baselineskip
\thispagestyle{empty}

\textbf{Mário de Andrade} (1893--1945) foi um dos nomes mais importantes do modernismo no Brasil. Além de poeta, romancista, historiador de arte e crítico, foi também um músico de formação erudita que, desde jovem, lecionou na área. Mas sempre atentou também para fenômenos musicais não-eruditos, que vão da música popular mais elementar, como uma cantiga ancestral de roda, ao mais elaborado, como a canção orquestrada para músicos através de partituras.

\textbf{A estranha força da canção} apresenta o pensamento de Mário de Andrade, com suas variações no decorrer do tempo, acerca da canção popular brasileira. O estudo da etnicidade nacional expressa na canção é feito a partir de diferentes perspectivas: ``Ensaio sobre a música popular brasileira'' e ``Gravação nacional'' oferecem, respectivamente, um panorama geral do assunto, e uma observação da indústria fonográfica e os
ritmos e autores que privilegiam. Já ``A pronúncia cantada\ldots'' se propõe a entender o problema a partir de uma análise linguística do cantar brasileiro. Por fim, em ``Dicionário musical brasileiro'' encontram-se alguns dos principais termos da música brasileira, com todos os vocábulos próprios e características históricas curiosas.

\textbf{Luís Augusto Fischer} é professor titular de Literatura Brasileira no Instituto de Letras da \textsc{ufrgs}, onde leciona desde 1984. Em 1992, criou um curso optativo de Canção Popular Brasileira para alunos de Letras, Ciências Humanas e Música, que funciona desde então. A partir dele, criou-se uma abertura para estudos e pesquisas de pós-graduação, onde tem orientado, com trabalhos realizados já há duas décadas. Articulou a criação do Núcleo de Estudos da Canção, junto à Pró-Reitoria de Extensão da \textsc{ufrgs}, que mantém programação desde então, com palestras, depoimentos e debates entre músicos, cancionistas, estudiosos e interessados. Junto a Guto Leite, organizou o livro \textit{O alcance da canção} (2016), que reuniu uma série de estudos realizados no Instituto em torno do tema. É autor de uma série de ensaios, artigos e resenhas sobre o universo da canção, publicados em jornais, revistas e livros.

\endgroup
\pagebreak
