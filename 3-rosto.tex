% Tamanhos
% \tiny
% \scriptsize
% \footnotesize
% \small 
% \normalsize
% \large 
% \Large 
% \LARGE 
% \huge
% \Huge

% Posicionamento
% \centering 
% \raggedright
% \raggedleft
% \vfill 
% \hfill 
% \vspace{Xcm}   % Colocar * caso esteja no começo de uma página. Ex: \vspace*{...}
% \hspace{Xcm}

% Estilo de página
% \thispagestyle{<<nosso>>}
% \thispagestyle{empty}
% \thispagestyle{plain}  (só número, sem cabeço)
% https://www.overleaf.com/learn/latex/Headers_and_footers

% Compilador que permite usar fonte de sistema: xelatex, lualatex
% Compilador que não permite usar fonte de sistema: latex, pdflatex

% Definindo fontes
% \setmainfont{Times New Roman}  % Todo o texto
% \newfontfamily\avenir{Avenir}  % Contexto

\begingroup\thispagestyle{empty}\vspace*{-.01\textheight}\parindent=0pt 
              \formular
              \Huge 
              \textbf{Mário de Andrade}\baselineskip=.67\baselineskip 

              \smaller[3]\textit{A estranha força da canção}
              \vspace{15mm}
              
              \LARGE
              Marcos Lacerda
              
              \vspace{5cm}

              \newfontfamily\minion{Minion Pro}
              %{\selectfont\minion\small
              %(\textit{organização})}
              
              {\selectfont\minion\footnotesize
              1ª edição}
                    
              \vfill

              \newfontfamily\timesnewroman{Times New Roman}
              {\fontsize{30}{40}\selectfont \timesnewroman hedra}
              
              \medskip

              {\selectfont\minion\small
              São Paulo \quad\the\year}
\endgroup
\pagebreak

\begingroup 

\footnotesize\parindent0pt\parskip5pt\thispagestyle{empty} 
\vspace*{.1\textheight}\mbox{} \vfill
\baselineskip=.92\baselineskip
\thispagestyle{empty}

\textbf{Contos e novelas} é uma antologia de narrativas
curtas de Júlia Lopes de Almeida, extraídas de duas de suas obras:
\emph{Ânsia eterna} (1903) e \emph{A isca} (1922). Da primeira, fortemente influenciada pelo escritor francês Guy de Maupassant, foram selecionados dez contos, marcados pelo insólito e pelo fantástico. Da segunda, que reunia originalmente quatro novelas, foram selecionadas duas
que apresentam algumas das características da narrativa de Júlia Lopes e dos temas que permeiam sua obra. Com tintas do naturalismo e do realismo francês, sua prosa tem traços da
objetividade, do antropocentrismo e do cientificismo que fizeram escola no século
\textsc{xix}. Não ficam de fora, no entanto, as críticas à sociedade brasileira:
o lugar da mulher na sociedade patriarcal, os conflitos familiares, as marcas da escravidão e os contrastes sociais, políticos e econômicos resultantes da modernização são temas recorrentes.

\textbf{Júlia Lopes de Almeida} (Rio de Janeiro, 1862--\textit{id.}, 1934) é uma das escritoras brasileiras mais importantes da virada do século \textsc{xix} para o \textsc{xx}.
Romancista, contista, cronista e dramaturga, publicou seus primeiros textos aos dezenove anos em jornais cariocas. Em 1886 mudou-se para Lisboa, cidade de seus pais, onde efetivamente inciou sua carreira de escritora. Seu primeiro romance, \textit{Memórias de Marta}, foi publicado em Portugal em 1888. Um dos principais nomes da \textit{Belle Époque}
carioca, Júlia Lopes publicou dez romances --- dentre os quais o famoso \textit{A falência}, de 1901---, cinco livros de contos e sete peças teatrais, muitas das quais escritas durante sua estadia na França.
Apesar de sua importância, foi pouco lida
se comparada aos escritores, em face da invisibilidade sofrida pelas
escritoras. Esteve também entre os idealizadores da Academia Brasileira de
Letras, mas foi preterida a assumir uma das cadeiras entre os fundadores
por ser mulher.

\textbf{Rodrigo Jorge Ribeiro Neves} é doutor em Estudos de Literatura e mestre em Letras pela Universidade Federal Fluminense (\textsc{uff}). Foi pesquisador visitante na Princeton University (\textsc{eua}) e bolsista da Fundação Casa de Rui Barbosa. Atuou como docente de literatura brasileira na \textsc{uff} e na Universidade Federal do Rio de Janeiro (\textsc{ufrj}). Desenvolveu pesquisa de pós"-doutorado no Instituto de Estudos Brasileiros da Universidade de São Paulo (\textsc{ieb"-usp}) e na Universidad de Alcalá, Espanha.

\textbf{Coleção Metabiblioteca} foi pensada para edições anotadas ou obras completas de cânones da literatura em língua portuguesa. As edições propõem desde estabelecimento de textos até novas hipóteses de leitura.

\endgroup
\pagebreak