\chapter*{Introdução\smallskip\subtitulo{O que chamamos de canção\\popular brasileira?}}
\addcontentsline{toc}{chapter}{Introdução, \textit{por Luís Augusto Fischer}}

\begin{flushright}
\textsc{luís augusto fischer}
\end{flushright}

\noindent{}É relativamente fácil esboçar uma súmula sobre o pensamento de Mário de
Andrade acerca da canção popular brasileira. Fácil e enganoso. Mas, espera um pouco: precisamos esclarecer de que se trata. O que chamamos de \textit{canção popular brasileira}?

A definição positiva é óbvia. Trata-se daquela composição que envolve
música e letra, em ligação íntima a ponto de qualquer das partes perder
muito de sua força ao ser considerada isoladamente, que dura uns três
minutos e costuma ser gravada --- aliás, sua existência depende
fortemente da existência de meios de gravação e de um mercado em que é
vendida. Em sentido mais sutil, a canção aqui considerada (há outras,
como a canção erudita e a canção da tradição folclórica) expressa o
ângulo de visão de um indivíduo, sua experiência pessoal, podendo por
isso ser considerada uma obra de arte moderna. Não modernista, calma
lá: moderna quer dizer aquela em que está clara a posição do indivíduo
criador em relação a sua criação; quer dizer aquela que pode --- e quer ---
ser vista como expressão de uma subjetividade, não como diversão
comunitária ou como mera realização de regras pré-existentes. Moderna
significa contemporânea do estado burguês, do Romantismo, da economia de
mercado, da hegemonia da cidade sobre o campo. Moderna quer dizer com
espaço para (e desejo de) invenção, ousadia, originalidade, conservando
sempre uma clara capacidade de comunicação com o ouvinte, que entende o
que ela diz.

Exemplos não faltam. É a obra de Noel Rosa, Ary Barroso, Ismael Silva,
Lupicínio Rodrigues, Adoniran Barbosa, Dorival Caymmi; ou, uma geração
depois, a obra de Tom Jobim, Caetano Veloso, Chico Buarque, Paulinho da
Viola, Rita Lee, Roberto Carlos. Poderíamos falar de outros países: para
a primeira geração mencionada, Carlos Gardel (quase sempre com um
parceiro, como Alfredo Le Pera) na Argentina, e Woody Guthrie nos \textsc{eua};
na geração posterior, a lista é infinita, com gente do quilate de Paul
Simon, Paul McCartney, Paul Anka e Pablo Milanés, para ficar apenas em
xarás do nosso genial Paulinho da Viola. 

Pode-se fazer também algumas definições negativas. A primeira: 
não é música feita para orquestra sinfônica; não é, quase
nunca, composta por gente que saiba escrever na partitura, e, aliás,
muitas vezes nem mesmo sabe escrever em português culto; não é portanto
música que se chama de \textit{clássica}, ou \textit{erudita}, ainda que em tempos
recentes haja concertos de orquestras dedicados à obra de cancionistas
populares. Aliás, a canção popular quase nada tem a ver com os
\textit{lieder}, como se denominam, em alemão, as canções eruditas, cantadas
por cantoras formadas em conservatório e com acompanhamento de
instrumento, especialmente o piano. A regra neste caso é tomar
um poema erudito e compor uma melodia para que ele seja cantado, bem
diferente da generalidade da canção popular que vamos comentar, cujo
nascimento é, muitas vezes, algo que desde sempre equilibra letra e
melodia mediante a entoação. A canção popular que vamos discutir é obra
para ser cantada em geral com singeleza, pouca instrumentação e voz
autoral, muitas vezes sem qualquer educação formal em canto. É banquinho
e violão, ou guitarra, baixo e bateria, ou cavaquinho e pandeiro, por
aí.

Mas a canção popular de que se vai falar aqui também não é --- segunda
definição negativa --- a canção popular anônima, ou de criação coletiva,
ou da tradição folclórica, como uma cantiga de roda, ou um canto de
trabalho. Esses casos são também canções no sentido de que se trata de
peças que juntam letra e melodia de modo íntimo e indesmanchável ---
``Atirei o pau no gato'', ``Ciranda, cirandinha'', etc.; esta canção aqui não é
algo criado por algum indivíduo, e por isso mesmo não expressa um ponto
de vista pessoal, não canta uma dor particular nem exalta uma alegria
com nome e sobrenome. Essa canção aqui se pode chamar de canção
folclórica, tradicional ou anônima (cada uma dessas qualificações
tem problemas e pode ser disputada), ainda que venha a ser gravada.
Tendo em vista esse tipo de canção, alguns autores pensarão na canção
popular brasileira que vamos comentar aqui como \textit{música popular
urbana}, sugerindo que a canção folclórica é rural ou ao menos nasce
fora das cidades. É outra designação insuficiente.

Última preliminar conceitual: pode haver trânsito entre esses três
modelos, naturalmente. Um cancionista popular moderno como Dorival
Caymmi compõe canções, como a ``Maracangalha'', que nem parece ter sido
composta por um indivíduo, nem parece ser moderna: ao ouvi-la, parece
que estamos entrando em contato com uma forma polida pelas gerações,
passada de uma para outra pela tradição oral, perdendo-se sua origem no
escuro dos tempos. Mas não: ela foi composta por um indivíduo singular,
com nome, endereço e \textsc{cpf}.

Na outra ponta, uma canção erudita pode vir a tornar-se popular e
assobiável, ou vir a fazer parte de uma canção popular gravada; da mesma
forma, um compositor acostumado a criar para orquestra pode tomar um
elemento de música folclórica e desenvolver toda uma ideia a partir
dele, como fizeram gênios feito Villa-Lobos. Há casos-limite, como a
obra de Arrigo Barnabé, que ousou compor canções de disco e de show, com
certa capacidade de circular como canção popular, mas tramadas segundo
pautas e exigências eruditas. Há experiências-limite, como o samba
``Sinal fechado'', de Paulinho de Viola, nascido, diz o autor, da
prática de um estudo de Bach.

Seja como for, vamos falar aqui do típico: a canção popular,
especificamente brasileira, esta que tem uns três minutos, nasce e se
fixa numa simbiose de letra e música, é gravada e é autoral --- e tem
expressado a experiência cultural no nosso país de forma marcante, a
ponto de podermos dizer que é ela que nos educa, que nos ensina a ser o
que somos e a sentir o que sentimos.

\section{síntese fácil e enganosa}

É relativamente fácil esboçar uma súmula sobre o
pensamento de Mário de Andrade sobre a canção popular brasileira. Fácil
e enganoso.

\textit{Fácil}. Ele não viu com bons olhos a existência dessa canção como uma
forma culturalmente válida ou, menos ainda, uma forma relevante na
cultura brasileira. Para ele, o valor mesmo, no campo das variadas
manifestações musicais, repousava em dois outros setores: a música de
orquestra --- erudita, clássica, de alto repertório --- e a música folclórica --- étnica, tradicional, comunitária, de autoria conhecida.

Mas é \textit{enganoso}. Já pelos dubitativos parênteses acima, bem como pelas
hesitações que revelam a existência de um quase pântano conceitual, uma
área de areia movediça, uma zona de águas turvas, podemos entrever que é
complicada a relação entre as palavras/\,conceitos e as realidades que
nelas e neles deviam estar descritas. E enganoso também porque Mário não
teve o relativo benefício de conhecer, como nós hoje conhecemos, o
alcance que a canção popular veio a ter no Brasil, muito especialmente a
partir da bossa nova, com o nascimento de sucessivas gerações de
talentos indiscutíveis de cancionistas, de que citamos exemplos logo
acima.

Também enganoso é, por um outro lado ainda, porque Mário escreveu muito
e dispersamente sobre todas as modalidades de música acima mencionadas
--- a popular, a folclórica e a erudita. Para complicar mais,
ou melhor, para matizar mais, ocorre que ele não manteve exatamente os
mesmos pontos de vista ao largo dos mais de trinta anos em que escreveu
sobre o tema, o que se pode verificar, como faremos aqui, em algumas
revisões que o autor promoveu em textos seus. Para flagrar o que
exatamente ele pensava será preciso antes demarcar com precisão o
momento em que a opinião foi emitida.

Hora então de recusar essa frase, essa síntese fácil e enganosa. De
tentar o caminho mais lento e mais justo. A pergunta que guia toda esta
apresentação é simples: o que pensava Mário ou, mais precisamente, o que
escreveu Mário sobre o que hoje chamamos sem maior dificuldade e com
relativa clareza conceitual de canção popular brasileira?

\section{a história pessoal de mário}

Mário viveu entre 1893 e 1945. Sobre o tema da música, como dito
acima, publicou muito --- mas dispersamente. Foi um músico de formação erudita
e, desde jovem, de 22 para 23 anos, foi professor na área. 
Manteve acesa por toda a vida a atenção para os fenômenos musicais
não eruditos, que vão do popular mais elementar, como uma cantiga
ancestral de roda, ao mais elaborado, como a canção gravada em discos,
eventualmente orquestrada para músicos que sabiam ler partitura, vendida
com sucesso no mercado e validada pelo assobio anônimo. Entre os vários
pontos extremos desse quadro --- o polo erudito e o popular, tradição
escrita e tradição oral, instrumentos de concerto e instrumentos de
batuque, campo e cidade, sagrado e profano, individual e grupal ---,
mundos inteiros se apresentam e simbolizam a experiência humana em seus
múltiplos e contraditórios aspectos, e da mesma forma mundos inteiros se
cruzam, se inseminam, se realimentam, por variados caminhos e com
diferentes intensidades.

Usemos a régua da história da canção no Brasil para pensar no caso
concretamente. Em seu tempo de vida, Mário primeiro ouviu discos
gravados em sistema mecânico, que em nosso país começaram a aparecer em
1902 --- foi então que se definiu a duração aproximada de três minutos para
a canção gravada, porque era o que cabia em cada gravação ---, e de 1927
em diante em sistema elétrico. Quase alcançou a gravação em sistema de
alta fidelidade na reprodução de sons gravados, que de fato só depois de
1945 se tornou acessível no mercado. Ouviu então aquelas gravações
precárias dos primeiros anos do século \textsc{xx}, que captavam mal e mal
reproduziam os sons mais complexos, de conjuntos instrumentais como os
de chorinho a bandas e orquestras, para depois de 1927 deliciar-se com as
sutilezas de interpretação tornadas possíveis com o advento do
microfone, do alto-falante e correlatos. Ouvia a ruidosa gravação da voz
roufenha do bahiano, e passou a ouvir a delicada interpretação da voz do
Mário Reis.

Era um adulto jovem quando o rádio começou a funcionar no Brasil, em
caráter ainda precário no \textit{ano-chave} de 1922, para se converter com o
tempo no veículo de difusão de música em geral, e da canção popular em
especial --- fala-se na \textit{era de ouro} do rádio justamente em referência
aos anos entre 1932 e 1950, mais ou menos. Isso quer dizer que Mário viu
nascer e desabrochar a geração de Noel Rosa, Ary Barroso, Ismael Silva,
Wilson Batista, Braguinha, Lamartine Babo, Cartola. Ouviu essa gente
toda até que a morte o colheu, ainda jovem, em 1945, aos 51 anos de
idade.

Terá tido tempo de ouvir a ``Aquarela do Brasil'', lançada em 1939, um
marco novo no processo já maduro de validação do samba carioca --- que
passou então de sua forma mais sincopada e batucada com o padrão do
Estácio, por sua forma um tanto amenizada, cancionalizada, em Noel Rosa,
até o samba-exaltação de Ary Barroso, samba para turista se embasbacar.
A ``Aquarela do Brasil'' foi gravada com um arranjo orquestral
requintado, concebido por um músico de formação erudita exigente,
Radamés Gnattali, sem a batucada dos músicos empiristas do morro e do
mangue. Mário teve o benefício de conhecer e ver desenvolver-se a obra
de uma geração decisiva para a canção popular no país.

Mas há outro exercício de datação que pode resultar interessante. Que
coisas, que figuras, que conquistas Mário não conheceu, no campo
que hoje chamamos serenamente de canção popular brasileira, figuras e
conquistas que para nós são o ar que respiramos, já tão integradas e
internalizadas na produção cancional no Brasil? Há quatro
momentos/\,movimentos que foram vanguarda mas também expressam mudanças
tectônicas, profundas, massudas.

O primeiro: a bossa nova, uma modernização na forma da canção brasileira
que a levou a ser conhecida mundo afora. O segundo: o significativo
aporte letrado ao plano da rotina da canção no país (e não só em nosso
país), em mais de uma geração, desde antes da bossa nova, passando pelas
décadas de 1960 e 1970 e alcançando o século \textsc{xxi}. O terceiro: a vigorosa
invenção da Tropicália, a feroz e libertadora geleia geral que tem se
mostrado capaz de capturar cenas brasileiras como poucos outros
discursos. O quarto elemento: o rap e sua impressionante capacidade de
expressar e simbolizar, no plano da canção, a experiência vital de
pobres, pretos e periféricos, a partir especialmente do trabalho dos
Racionais.

O que Mário teria dito de cada um desses momentos? Como teria acolhido
essas fortes mudanças no campo da canção em seu quadro conceitual? Que
alterações esse quadro teria sofrido caso Mário os tivesse conhecido?
Jamais saberemos, é claro. Trata-se de especulação, que nos ajuda mesmo
assim a localizar historicamente as visões do autor sobre a canção
popular brasileira.

Acrescentemos mais um patamar de complexidade: Mário de Andrade se associou, 
segundo mais de uma maneira, às pesquisas
do folclore --- e \textit{folclore} é outro tema fugidio. Termo inventado em
1845, pela junção de \textit{folk},\footnote{Em tradução do inglês, \textit{povo}.} e \textit{lore},\footnote{Também em tradução do inglês, \textit{conhecimento}.}
significa originalmente o conjunto de histórias transmitidas
oralmente por um povo --- de um lugar de fora das grandes
cidades, que tende a homogeneizar registros culturais
anteriores e extraviar singularidades, fazendo-as cair no grande
caldeirão da cultura de massas que no qual a tendência é transformar tudo em
mercadoria ---, o folclore se forjou numa zona que fazia fronteira ou
tinha intersecção com campos do saber já bastante nítidos, como a
história, e com outros em processo de invenção ou consolidação, como a
sociologia e a antropologia, assim como com outras disciplinas que na
Europa contavam com carreiras muito nítidas, como a filologia (o nome
antigo dos estudos literários) e a música.

Ao nomear todos esses campos, já se vê o tamanho da encrenca conceitual
implicada no nome \textit{folclore}. Mas é pior ainda. No Brasil e noutras
partes, especialmente da América, os anos de 1920 a 50 são o tempo da
estruturação das modernas áreas universitárias de humanidades, letras e
artes --- e afinal, onde entraria o folclore nisso tudo? Seria um
departamento, como antropologia? Ou seria uma parte da sociologia? Um
assunto a mais na preparação de professores de música? A filologia
deveria meter sua lente no estudo das coisas tidas como folclóricas?

Todas essas alternativas estavam no horizonte e foram se configurando
variadamente, conforme as circunstâncias de cada cidade ou país. Para
além delas ainda outra dimensão se sobrepôs: criada a \textsc{onu}, em 1945, logo
foi organizada a seção de Educação, Ciência e Cultura, a \textsc{unesco}, que em
1946 tratou de recomendar aos países-membros que organizassem \textit{seções
de folclore} em toda parte, como forma de proteger ou ao menos de
salvar do soterramento várias modalidades de saber e de lazer que
correspondiam a formas de convivência social pré-modernas, pré-mercado,
de transmissão oral, de permanência incerta porque dependente da memória
coletiva, da intuição, da vida comunitária. Mário não viveu para ver
essa institucionalização, mas em seu tempo de vida foi, ele mesmo, um
agente de colecionismo folclorista, quando produziu e protagonizou
missões de coleta de material folclórico (especialmente nos anos 1927,
28 e 29, com viagens ao Norte e ao Nordeste do país) e dirigiu o
Departamento de Cultura da capital paulista (em 1936 e 37).

Aquilo que nos anos 1940 podia ser chamado serenamente de \textit{fato
folclórico}, como um cantoria comunitária, um joguinho infantil feito
com ossos de animais ou uma técnica de costura, passou a ser disputado,
por assim dizer, por várias áreas do conhecimento e da prática social.
Criaram-se grupos de prática organizada de coisas que a disciplina
folclórica recolheu da dispersão da vida vivida. Grupos que pareciam
tranquilos ao considerar o fato folclórico como parte de uma tradição
assumida como verdadeira para a região --- são sociedades de canto e
dança tradicionais, centros de tradição, exposições para venda de
produtos de fabricação até ali espontânea e comunitária, etc. ---; nos
estudos acadêmicos, uma problematização intensa para averiguar se essa
matéria-prima merece ser abordada numa área autônoma ou deve ser alocada
em alguma das ciências humanas. Uma geração depois da morte de Mário, se
estabelece uma zona de fronteira compartilhada entre Música e
Antropologia que vai se chamar \textit{etnomusicologia}, que lida, por
exemplo, com práticas musicais não escritas, populares, coletivas, etc.,
como as congadas, o jongo ou certas modalidades de samba. Mário, se
vivo, teria virado um etnomusicólogo, provavelmente.

Isso para nem falar de intensos, renovados, frutíferos trânsitos entre
esse mundo musical, popular e espontâneo, e a produção de canções e de
música em geral para o mercado, desde o século \textsc{xix}, no tempo da impressão
de partituras, até hoje, na era da internet e dos formatos digitais.
Quando Caetano Veloso grava, em formato digital, um samba de roda que
originalmente era cantado e dançado comunitariamente, e essa gravação é
vendida e permite reprodução em outros ambientes, digamos numa festa
urbana de gente de formação universitária --- o que está acontecendo? Que
conceito pode abranger e descrever esse caso? O samba original deixou de
ser o que era? Se transformou? Perdeu algo? Ganhou algo?

\section{a canção em três momentos}

Dado esse quadro histórico aproximativo, é hora de mergulhar nas
concepções e impressões que Mário de Andrade escreveu. Como dito antes,
o material é vastíssimo, e parte dele vai estampada neste volume.
Naturalmente, um estudo introdutório como este não pode pretender a
exaustividade --- há teses de alto valor abordando a visão de Mário sobre
música, como por exemplo ``Da música folclórica à música mecânica ---
uma história do conceito de música popular por intermédio de Mário de
Andrade'' (1893-1945), de Juliana Pérez González.%\footnote{\textsc{usp}, 2012.}

O que vai adiante é um mapeamento da visão de Mário sobre a canção
popular brasileira, dividida em três seções. A primeira repassa alguns
dos mais importantes escritos do autor no tema até 1930 a 31, que
acompanha o problema até o momento inicial da carreira daquela geração
que fez o rádio e se fez no rádio, Noel Rosa por exemplo. Nesta fase se
destaca o famoso \textit{Ensaio sobre a música brasileira}, de 1928.

A segunda seção vai abordar os textos de Mário produzidos entre esse
limite inicial e os anos de 1940 e 41 --- neste ano Mário volta a viver
em São Paulo, depois de uma intensa experiência vivida na então capital,
o Rio de Janeiro, entre meados de 1938 e março de 1941. O Rio era,
então, o epicentro da vida musical popular no país, ou ao menos o centro
das relações entre criação cancional, gravação e divulgação via rádio e
shows. Há ensaios importantes produzidos nesse tempo, que representa um
período intermediário de produção e de conceitos, e na cronologia da
história da canção popular que nos interessa aqui temos a baliza
importante representada pela gravação já mencionada de ``Aquarela do
Brasil'', em 1939.

A última seção abordará os escritos e reescritos até o final de sua
vida, em fevereiro de 1945 --- antes do final da Segunda Guerra Mundial,
que pode ser datado de maio desse ano, quando os Aliados derrotaram
Hitler, evento que o motivou a reflexões sobre o papel da arte num mundo
assolado pelo nazismo, e antes da queda do Getúlio do Estado Novo, em
outubro daquele ano.

\subsection{definição pré-modernista}

No mencionado \textit{Ensaio sobre a música brasileira}, de 1928, Mário
trabalha com um par de conceitos opostos que até então parecem
suficientes para dar conta do mundo da música. De um lado, há a \textit{música
artística}, quer dizer, erudita; de outro, a \textit{música popular}. A
primeira vem caracterizada no ensaio como \textit{imediatamente
desinteressada}, enquanto a segunda é \textit{interessada}, ou \textit{de
circunstância}. Nessa oposição se expressa uma diferenciação corrente
no final do século \textsc{xix}, naquele tempo posterior ao Romantismo que, no
Brasil, se tornou conhecido pela extraordinária força do Parnasianismo,
que propugnava ``a arte pela arte'', ou seja, justamente essa \textit{arte
desinteressada}. É uma definição pré-modernista, ou, para evitar a
confusão com o termo que no Brasil se usa como óbvio, o Modernismo (que
não é nada óbvio), uma definição anterior às vanguardas estéticas que
povoaram o Ocidente, do Simbolismo em diante.

A ideia de que a \textit{melhor} arte seja qualificável como
\textit{desinteressada} é, por si, a expressão de um conceito que seria
derrubado vivamente pelas vanguardas e pelo século \textsc{xx} em geral: para
essa visão, reiterada aqui por Mário de Andrade, a boa arte não pode ter
interesses para além de sua própria expressão, sua mera existência: não
pode expressar uma visão do mundo, não pode expressar uma ideologia, uma
utopia, nem pode pretender por exemplo ter valor de mercado. Veja-se que
neste espectro praticamente não há espaço para o que nós hoje chamamos
de canção popular, que nem aparece aqui.\footnote{Em 1945, ao final de sua vida, Mário dirá quase o contrário: defenderá
o engajamento da arte, quer dizer, a produção de arte interessada, na
conjuntura de combate ao nazismo, como veremos adiante.}

A solução para as dificuldades da criação musical brasileira estaria,
para o autor, em encontrar-se com o Brasil, com a brasilidade. ``Uma
arte nacional já está feita na inconsciência do povo'', ele afirma. A
ordem dos fatores então deve ser: 

\begin{enumerate}
\item Atenção ao povo;
\item Encontro com a brasilidade;
\item Verdadeira criação nacional.
\end{enumerate}

Nessa sequência, estaria
resolvida a distância entre as duas modalidades de música: ``O artista
tem só que dar pros elementos já existentes uma transposição erudita que
faça, da música popular, música erudita, isto é: imediatamente
desinteressada.'' O processo então seria simples: há uma forma
excelente, insuperável, que é a música erudita ou artística, aquela que
é desinteressada; na direção desse ideal é que a música popular (que é
interessada, de circunstância) deve ser conduzida pelo artista que
souber auscultar a inconsciência do povo, onde repousa a arte nacional.

Visto de hoje, o ensaio guarda complexidades meio estranhas. Não é que
Mário tenha preconceito contra alguma modalidade de música. Logo no
início do ensaio ele passa pelo padre José Maurício e por Carlos Gomes e
Villa-Lobos, mas também pelos Oito Batutas, o grupo liderado por
Pixinguinha que teve enorme papel na divulgação do chorinho, no Brasil e
em outros países, e por Sinhô, talvez o primeiro compositor popular a
usar a palavra \textit{samba} para definir-se --- ele foi o primeiro a se
qualificar como ``rei do samba''. O ponto problemático, no plano
conceitual, parece estar ainda na relação do Brasil com a Europa. ``Até
há pouco a música artística brasileira viveu divorciada da nossa
entidade racial'' é a frase que abre o ensaio.

Como se pode deduzir, então, é ainda uma briga com a Europa, uma luta
pela diferenciação do nosso país em relação com a Europa, num eco
perfeito da longa jornada modernista paulista em busca de sua afirmação
e conquista da hegemonia nacional --- que foi obtida, com o tempo. A
proposta, nos anos iniciais, era encontrar uma nova definição de
nacionalidade, uma nova síntese que representasse a ``raça brasileira'',
segundo a denominação usada por Mário, que ao mesmo tempo fosse
comparável com aquela configurada com o romantismo de Alencar, escritor
que era uma confessa admiração de Mário, e contrastável com o que
parecia ser o abandono do nacionalismo por parte dos escritores da
capital nacional na virada do século, como os parnasianos e mesmo como
Machado de Assis, tão pouco nacionais na visão de Mário.

No campo específico que aqui interessa, a averiguação dos conceitos do
autor quanto ao mundo da canção popular, a figura de maior relevo parece
ser a de Sinhô.\footnote{João Barbosa da Silva, \textsc{rj}, 1888--1920.} No auge de seu
prestígio em 1928, com gravações, polêmicas, sucesso e mesmo com
prestígio entre letrados inventivos --- em 1929 ele viajaria a São Paulo,
a convite dos \textit{antropófagos} de Oswald de Andrade ---, Sinhô é visto
por Mário numa posição singular: ``Os maxixes impressos de Sinhô são no
geral banalidades melódicas. Executados, são peças soberbas, a melodia
se transfigurando ao ritmo novo.''

O que temos aqui nesta declaração é precioso: ao mesmo tempo que Mário
vê banalidade melódica, percebe originalidade rítmica, e aqui estaria
uma das riquezas realmente fortes da música brasileira. Quanto à
expressão ``maxixes impressos'', vale uma análise: primeiro, \textit{maxixe}
era inicialmente o nome de uma forma de dançar aquele tipo de música que
não era mais a polca, nem era o puro batuque, e agora era um gênero
musical, ainda não a síntese nova do samba carioca, nem o chorinho
estável. O nome da dança passou a designar o estilo musical, de que
``Jura'', de Sinhô, é exemplo ainda hoje vivo. Já o \textit{impresso} quer
dizer o escrito, o fixado na página, que resulta frio e banal. De certa
forma, Mário parece aqui estar marcando a posição limítrofe de Sinhô,
não apenas entre a forma fria do impresso e a forma quente da
performance, mas também entre o cerebral, escrito, ligado ao campo
erudito e, na outra ponta, o corporal, espontâneo, mergulhado no campo
iletrado, oral, popular. A mediação entre os dois lados se faz, já em
1928, pelo impresso e pelo gravado, na partitura e no disco.

Dá para perceber uma certa ambivalência de Mário de Andrade? Pois é. Os
dois polos em que dividia a música --- a alta, desinteressada, artística,
erudita, e a baixa, de circunstância, popular --- já não eram mais
suficientes para acolher essa forte novidade, justamente a canção
popular de um Sinhô, que certamente não atuava no campo erudito, mas
também não podia ser enquadrado no outro campo, aquele em que se
localizavam as canções ditas folclóricas, de matriz rural e circulação
comunitária. A simpatia de Mário para um caso como o de Sinhô faz par
com outro argumento seu, neste ensaio de 1928: a ``destruição do
preconceito da síncopa'', porque nela estava uma marca essencial da
criação musical popular no Brasil --- e ``a música popular brasileira é a
mais completa, mais totalmente nacional, mais forte criação da nossa
raça até agora''.

Essa segurança dava ao estudioso a sensação de que era preciso
investigar bem o fenômeno, para entender o processo que tinha, no fundo,
o mesmo sentido geral do programa modernista que sua turma levou
adiante, a saber, a redefinição da identidade do país, segundo
parâmetros novos. Se era verdade que havia muitas influências sobre a
criação musical entre nós --- do mundo espanhol ou hispânico, com a
habanera e o tango, das danças europeias populares, com a valsa, a
mazurca, a polca, e do flamante jazz norte-americano ---, também era
certo que as matrizes lusas, africanas e ameríndias tinham solidez na
formação da cultura brasileira.

Mas persistiria, até o fim de sua vida, a hesitação de Mário em aceitar
plenamente a canção popular autoral gravada e comercializada como um
produto válido da cultura brasileira. Hesitação que se devia em parte,
talvez, a certo preconceito dele para com a música que encontrava
mercado entre os meios modernos de gravação e difusão, como o disco e o
rádio, percepção que se completava com a fantasia de que a arte erudita
e a arte popular estavam acima e fora do mercado, a primeira porque era
\textit{desinteressada} e sublime, a segunda porque vinha da fonte pura do
povo anônimo e iletrado.

A certa altura do ensaio, Mário chega na vizinhança de um novo conceito,
que seria capaz de designar essa outra modalidade de canção, como a de
Sinhô, de que ele gostava e não gostava, e que não era mais nem erudita
nem folclórica. A designação é ainda tateante: ``Na cantiga praceana o
brasileiro gosta de saltos melódicos audaciosos de sétima, de oitava'',
como se ouvia na criação de Chiquinha Gonzaga, ``e até de nona que nem
no lundu 'Yayá, você quer morrer', de Xisto Bahia.''

\textit{Praceana} quer dizer \textit{da praça}, isto é, da cidade, urbana, ali naquele
caldo grosso de cultura em que tudo circula e se mistura, perdendo a
nitidez daquela divisão tão sólida com que Mário lidava (mas que não
dava mais conta da realidade) mas ganhando justamente a novidade, o
frescor da criação, que tanto futuro teria, no rastro das Chiquinhas e
Sinhôs. A palavra \textit{praceana} trai a ambivalência de Mário ao lidar com
a canção popular: quem diz praceana está dizendo ao mesmo tempo
\textit{urbana} como mera designação geográfica e, na direção contrária,
sugere que é coisa submetida ao comércio da praça, coisa vulgar,
rebaixada.

Em artigo de 1930, ``Gravação nacional'', Mário discute o valor
etnográfico dos discos. Sua pergunta inicial poderia ser escrita da
seguinte maneira: podemos confiar nos discos que a indústria tem
colocado no mercado como fonte de documentação etnográfica? Vista agora,
a pergunta tem um quê de descalibrado: por que o valor de um disco
estaria em ser documento de estudo de folclore, afinal? Mas o simples
fato de que Mário formule o problema revela a lente pela qual filtrava a
novidade da canção popular urbana, \textit{praceana}, que ele obrigava a
caber nas duas polaridades acima mencionadas, ou bem como arte erudita e
desinteressada, ou bem como arte popular e espontânea.

Mas a \textit{praceana} não cabia bem na disjunção. E mesmo assim Mário não
hesitava em manter a lente em funcionamento. Daí sua lamentação de que o
disco brasileiro seja ``quase que exclusivamente do domínio da música
popular urbana'', ``banalizada pelos males da cidadania''. Em vez de
detectar a novidade cultural e aceitar sua força, Mário a tratava como
um incômodo. Atrás desse tratamento mal se escondia o preconceito do
crítico para com a cidade, o cadinho cultural urbano. Vamos ver que nos
dois momentos seguintes a visão do crítico vai sofrer alguma mudança
justamente nesse ponto. Estava começando a se apresentar no cenário a
geração de Noel Rosa, Ismael Silva, Ary Barroso, Geraldo Pereira e
tantos outros mais.

\subsection{a ciência do folclore}

A roda do tempo girou e Mário permaneceu atento. Por um lado, se
firmava, na década de 1930, o ensino universitário, que na \textsc{usp},
exemplarmente, acolhia a moderníssima ciência social chamada de
sociologia, em cujo âmbito havia lugar para o estudo das práticas
culturais populares, sob a rubrica do folclore. Seria este o lugar
para a canção popular que agora era veiculada no rádio? Por outro, pela
mesma época acontecia uma inesperada profissionalização no \textit{metiê}: gente
que até há pouco compunha sambas para se divertir com os amigos, para
namorar ou chorar as dores da saudade, em caráter amador, agora se via
envolvida pela lógica do mercado, podendo ganhar um inesperado dinheiro
com a antiga diversão, com o tempo alguns chegando a viver dos ganhos da
música (venda de discos, cachês, etc.), que ia ser assobiada porque ia
ganhar o coração dos ouvintes. Isso tudo ocorrendo nas cidades, aquele
ambiente que, vimos antes, Mário abominava, por causa dos \textit{males} que
impunha. E agora?

No estudo ``A música e a canção populares no Brasil'', de 1936, o
crítico paulistano tem mais cautelas do que antes. Seu foco, aqui, é o
que chama de a ciência do folclore: ``Tanto no campo quanto na cidade
florescem com enorme abundância canções e danças que apresentam todos os
caracteres que a ciência exige para determinar a validade folclórica
duma manifestação.'' Parecia haver segurança no estabelecimento dos
juízos, por parte dessa ciência.

Mas apenas parecia, porque na América as coisas não eram tão nítidas.
Por um lado, seria ``de boa ciência afastar-se de qualquer colheita
folclórica e documentação nas grandes cidade, quase sempre impura'' ---
quer dizer, a pureza estava no mundo rural. Mas essa restrição fazia
pouco sentido no Brasil: aqui, ``as condições de rapidez, falta de
equilíbrio e de unidade do progresso americano tornam indelimitáveis
espiritualmente, para nós, as zonas rural e urbana.'' Dá pra ver que
Mário, agora, está quase abandonando aquela polaridade simples, com arte
erudita estando para a cidade assim como arte popular estaria para o
campo. Não dava certo aqui: em regiões ricas, até mesmo pequenas cidades
do sertão já tinham ``água encanada, esgoto, luz elétrica e rádio''; em
cidades grandes como o Rio, Recife ou Belém, por outro lado, ``apesar de
todo o internacionalismo e cultura, encontram-se núcleos de música
popular em que a influência deletéria do urbanismo não penetra.''

A coisa definitivamente não era simples, e o que estava em jogo era
muito, como Mário sabia ou intuía. Daí sua conclusão acolhedora:
``Recusar a música popular nacional só por não possuir ela documentos
fixos, como recusar a documentação urbana só por ser urbana, é
desconhecer a realidade brasileira.'' Vitória dos fatos contra o
preconceito e os conceitos inadequados. Ou quase.

No ensaio ``Evolução social da música no Brasil'', de 1939, Mário dá
atestado da filiação de seu raciocínio à tradição acadêmica europeia,
porque ao acompanhar o tema que está no título --- evolução social
da música entre nós, com ênfase no adjetivo --- tenta mostrar o
nascimento e o florescimento da música popular brasileira a partir da
música religiosa da colonização portuguesa, acompanhando a cronologia da
história brasileira. Hoje se pode duvidar dessa filiação com mais
clareza, vendo na canção popular vitoriosa uma forma artística nascida
mais da entoação espontânea do que da música escrita, por exemplo ---,
mas para o estudioso paulista essa descendência era coisa dada.

Por esse viés é que ele repassa a trajetória da música como função da
Igreja Católica antes da Independência, e só depois desta podem aparecer
compositores (eruditos) como Francisco Manuel da Silva e Carlos Gomes,
produzindo uma música, diz ele, internacionalista. Mas a República viria
modificar a situação, livrando o país da monarquia em favor de maior
democracia. Aqui é criado o Instituto Nacional de Música, e aqui
aparecem figuras singulares, ainda do mundo erudito, como Alberto
Nepomuceno.

E no campo popular? Durante a Colônia, não houve música popular
brasileira, e sim criações em setores estanques: ``os negros faziam a
sua música negra lá deles, os portugueses a sua música portuga, os
índios a sua música ameríndia.'' No final do século \textsc{xviii} é que ``um povo
nacional vai se delineando musicalmente'', com formas discerníveis: ``o
lundu, a modinha, a sincopação'', mais as danças dramáticas populares
(reisados, cheganças, cabocolinhos, bumba-meu-boi, congados). No final
do Império, esse campo popular viu a modinha passar ``do piano dos
salões para o violão das esquinas'', viu se consolidarem o maxixe, o
samba, os conjuntos seresteiros de choros. Tudo isso, mais a ``evolução
da toada e das danças rurais'', conduz a uma conclusão bastante nova, em
se tratando do autor: ``a música popular cresce e se define com uma
rapidez incrível, tornando-se violentamente a criação mais forte e a
caracterização mais bela da nossa raça.''

Dá para perceber claramente que, neste ensaio, música popular é
aquela urbana, ou que se desenvolveu na cidade, e é aquela que
caracteriza mais que outras formas ``a nossa raça'', a raça brasileira,
que os modernistas julgavam estar finalmente se revelando, com certa
estabilidade. Claro que nisso ia muito de idealização, e não se trata
aqui de concordar com o autor; o ponto é ver que neste momento não há
mais aquele hiato conceitual entre o polo erudito e o polo folclórico,
hiato que era habitado justamente pela canção popular que nos interessa
aqui. Não: aqui, para ele, a canção popular é parte viva da música
popular, a mais bela caracterização dos brasileiros. Isso tudo,
registre-se, no contexto de um ensaio que aborda essencialmente a música
erudita!

Mas seria exagero considerar que Mário de Andrade houvesse em definitivo
acolhido a canção popular gravada e comercializada como um fruto maduro
e relevante da cultura brasileira. Nada disso. Ainda há ambivalências,
como se vai ver no ensaio ``Romantismo musical'', uma conferência dada
em 1941. Ainda nos começos do raciocínio, ele lembra uma ocasião passada
no Amazonas, onde estivera anos antes. Lembra que, o navio subindo pelo
rio em região ``sem homem branco ou seringais'', quer dizer sem presença
ocidental, ouvia ``frágeis mas penetrantes assobios humanos.'' Seriam
``tapuios semicivilizados'', conforme explicam a ele na ocasião, que com
os assovios se comunicavam contando da presença do navio em que o
estudioso ia. ``Eu escutava essa música\ldots{} romântica, simples conversa
entre tapuios.''

Essa experiência lembrou a ele os cantos de aboiar que ouviu no
Nordeste, cantos que também era comunicação de vaqueiro para vaqueiro,
de fazenda a fazenda. Mário então associa essas manifestações assoviadas
e cantadas à ``musicalidade das linguagens infantis e dos primitivos.''
E conclui que tudo isso era derivado de uma ``origem legítima'', ``uma
base biológica natural, o grito'', origem primeira dos sons
inarticulados e dos sons articulados, ``o Ré bemol e a palavra, a música
e o verbo.''

Há uma outra idealização aqui: Mário julga estar em contato com a gênese
do canto, e por aí na gênese da canção, da música cantada, que é o
centro do interesse da presente antologia e deste ensaio. Ele chega a
imaginar que se os primitivos humanos tivessem convencionado em sons os
significados das palavras mais necessárias, ``nós hoje estaríamos nos
comunicando uns com os outros por meio de árias e cantiguinhas, melodias
infinitas, hinos e até marchas totalitárias'' --- bem, não esqueçamos que
em 1941 havia Hitler, Stálin, o Estado Novo brasileiro e a Segunda
Guerra Mundial, temas que ocupam o horizonte mental do autor, como
adiante veremos.

No que interesse de perto aqui, interessa seguir o rumo dessa reflexão
até alcançarmos a canção popular. Ela de fato aparece em seguida, de
modo inesperado. Mário vai afirmar que o Romantismo ``era por princípio
popularesco'' --- e aqui entra em cena esse outro adjetivo,
\textit{popularesco}, que acrescenta complexidade e perplexidade ao quadro
conceitual com que o autor pensa sobre a canção popular. Quem diz
popularesco não diz popular: este deriva do povo, mas aquele sugere uma
relação forçada, talvez até uma usurpação de direitos. Estaria a canção
de três minutos, gravada e comercializada, aquela de Noel Rosa e
Lupicínio Rodrigues, em qual dos casos?

Escritores românticos europeus, lembra o ensaio, se interessaram por
poesias e cantigas populares; tal foi o caso do francês Chateaubriand,
tal foi o caso da fraude perpetrada pelo escocês James MacPherson ao
fajutar um poeta ancestral que escrevia em língua antiga. O francês
seria, diz Mário, ``um legítimo precursor do folclore, com sua obsessão
pelas canções escutadas na infância.'' Em certo sentido, podemos
concluir que o Mário de Andrade da cena inicial evocada, com sua
obsessão pelo assovio e pelo aboio, seria um caso redivivo do mesmo
Chateaubriand, ambos pensando em estarem em contato com a fonte da
cantiga, de certa forma.

Algumas páginas adiante encontramos o ponto mais sensível dessa
reflexão. É quando Mário afirma que o tempo do Romantismo era o tempo
``da cantarolagem'', da cantoria fácil e desabrida, do canto banal de
tão frequente. E dá exemplo muito significativo, que nos interessa
agora: ``Dalaire lastimava em 1845 não existirem mais editores de
quartetos ou sinfonias {[}composições eruditas para instrumentos de
orquestra{]}, ao passo que ninguém hesitava em dar seis mil francos por
seis romanças de compositor em voga, desde que elas fossem lançadas por
intérpretes como as senhoritas D'Hénin e Drouard, ou cantores como
Penchard, Vartel ou Richelmi.''

A romança era um gênero poético-musical, de caráter sentimental e com
origem medieval, que em última instância era o mesmo que viria a ser a
canção popular que nos interessa aqui: uma forma que mesclava música e
poesia de modo significativo para cantar amores e desamores. O que Mário
conta, evocando esse Dalaire, é que em 1845 era difícil publicar peças
musicais eruditas por escrito, ao passo que formas musicais cantadas
vendiam bem. Por outro lado, o crítico vincula a interpretação, por
parte dos cantores, à perecibilidade, e, na mão oposta, vincula a
escrita e a edição, coisa do autor e do editor, à permanência no tempo.
Se dermos mais um passo, Mário está aqui lidando com o paradoxo que está
no coração da música erudita: é escrita (e depois será gravada), e por
isso pode permanecer no tempo, mas sua natureza é ligada à performance,
e por isso morre assim que é enunciada.

Neste ponto do ensaio, ele aproxima vertiginosamente essa história do
Romantismo ao seu presente: ``Estes {[}cantores e cantoras{]} seriam por
certo os Orlando Silva e as Carmen Miranda do tempo, sem rádio nem
disco, predestinados à morte irremediável.''

Traduzindo: naquele 1941, também seria difícil vender partituras de
música erudita para quartetos ou outros conjuntos orquestrais, enquanto
os sucessos da hora, que se expressavam na canção popular, vendiam como
pão quente --- associados ao rádio e ao disco, portanto mergulhados no
pântano da indústria cultural. Então, conclui Mário, se destinam ``à
morte irremediável'' --- por serem vinculados à performance e por estarem
atrelados à venda, ao mercado. Mas morte? De quem? E por que essa
reserva (que é também má vontade) com os dois grandes cantores citados,
Orlando Silva e Carmen Miranda?

Mário mais uma vez mostra sua inconformidade com o que considera uma
inversão de valores --- a música erudita é que devia se eternizar, e a
seu lado também a música folclórica, mas não a música \textit{praceana}, e
por isso ele vaticina a morte dos dois popstars brasileiros do momento.
Mas o futuro não daria razão a ele.

\subsection{um estado dinâmico do ser}

Nos últimos anos de vida, Mário de Andrade acentuou alguns aspectos de
sua discussão sobre o papel da arte, especialmente da música, no mundo
em que vivia --- o mundo do auge da Segunda Guerra Mundial, do nazismo.
Muita água havia corrido desde o tempo em que começara a pensar sobre a
canção popular; a canção popular, ela mesma, quase nem se reconhecia
mais se se olhasse no espelho daquele tempo. Em lugar de Sinhô e seus
singelos maxixes, como em 1929, agora imperava gente como Ary Barroso,
gravado com arranjos eruditos de um Radamés Gnattali e outros maestros
de formação erudita.

As ideias e intuições do autor aparecem em quantidade. Num texto desses
anos,\footnote{Porém sem data precisa, reeditado em \textit{A música na vitrola de
Mário de Andrade}.} brilha a percepção de que a melodia da canção
popular não nasce de uma sequência de acordes, mas de ``um estado
dinâmico do ser'', ideia que se aproxima do que, três gerações depois,
Luiz Tatit vai postular como sendo a matriz da melhor canção popular, a
entoação, não a dinâmica de uma harmonia.

Em carta a Moacyr Werneck de Castro, amigo feito na temporada carioca,
datada de fevereiro de 1942, Mário declara gostar de ``Amélia'', de
Ataúlfo Alves e Mário Lago, samba por muitos motivos famoso pelos anos
afora, e de ``Praça Onze'', de Grande Otelo e Herivelto Martins, outro
samba de grande futuro. No contexto dessa declaração, cogita a
necessidade de alguém ``fazer um estudo sobre os textos do samba
carioca'', que com frequência é ``genial na necessidade de dizer as
coisas'', contendo ``uma riqueza psicológica assombrosa''.

Era uma espécie de reconhecimento do grande valor, inclusive literário,
desse gênero de música no país, certamente. E faz lembrar um comentário
de Jorge Luis Borges, que nada tinha de musical em sua formação, ao
comentar as letras do tango, em seu país. A analogia é imperfeita mas se
sustenta, não apenas porque o samba e o tango têm trajetórias históricas
muito assemelhadas, como pelo fato de que o Borges jovem era tão
interessado numa nova síntese identitária quanto Mário. Diz o grande
escritor portenho que ``as letras de tango {[}\ldots{}{]} integram, ao fim de
meio século, um quase inextricável \textit{corpus poeticum} que os
historiadores da literatura argentina lerão ou, em todo caso,
vindicarão.''\footnote{``História do tango'', em \textit{Evaristo Carriego}, 1930.}

Um flagrante nítido das oscilações conceituais de Mário acerca de nosso
objeto aqui se encontra na \textit{Pequena história da música}. O livro
teve uma primeira encarnação sob o nome de \textit{Compêndio de história
da música}, editado em 1929, portanto no primeiro momento de sua
trajetória de comentarista musical; a segunda, com o nome de
\textit{Pequena história da música}, saiu em 1944, no momento derradeiro
de sua vida. Uma comparação sumária entre as duas edições revela a
mudança em vários níveis, no que nos interessa aqui.

O capítulo \textsc{xi} da primeira edição se chamava ``Música artística
brasileira''; na segunda, ``Música erudita brasileira''. Na transição
entre os dois adjetivos vai, é de supor, o reconhecimento de que nem só
no campo erudito se encontra arte. Um ponto a favor do reconhecimento do
valor da canção popular.

%\textsc{xii},
No capítulo dedicado à ``Música popular brasileira'', a edição de
1929 ostenta um derradeiro parágrafo que assim começa: ``As
manifestações que
tiveram maior e mais geral desenvolvimento são, desde o século passado,
as Modinhas e os Maxixes que andam profusamente impressos.'' Na mesma
posição, a edição de 1944 diz: ``As manifestações \textit{popularescas} que
tiveram maior e mais geral desenvolvimento são, desde o século passado,
as modinhas e os maxixes e \textit{sambas urbanos} que andam profusamente
impressos'' --- as sublinhas ressaltam a diferença. Mário propõe um
adjetivo de tom depreciativo, \textit{popularescas}, para distinguir essa
modalidade de música, a canção popular tal como aqui definida, das
outras modalidades de canção que ele têm no horizonte, a canção erudita,
nítida em suas diferenças, e a canção popular, aquela de criação
coletiva, anônima ou desconhecida, ``autêntica'' em sua suposta pureza,
rural e fora do circuito do mercado. Um ponto contra a mesma canção
popular, agora chamada de ``popularesca''.

No mesmo parágrafo, entre a primeira e a derradeira edição ressalta
outra diferença, agora na designação de autores de sambas. Em 1929, o
autor menciona Eduardo Souto, Donga e Sinhô, ``as figuras contemporâneas
mais interessantes do maxixe impresso''; em 1944, na mesma posição do
texto encontraremos ``Donga, Sinhô e Noel Rosa, as figuras
contemporâneas mais interessantes do samba impresso.'' Sai Edmundo Souto
e entra Noel Rosa; sai o maxixe e entra o samba. Um ponto a favor da
canção, neste reconhecimento do valor de Noel, que de fato expandiu a
capacidade da canção em enunciar as experiências complexas da vida.

Finalmente, nesta comparação sumária entre as duas formas do mesmo
estudo, vale visitar o final do livro, ao cabo do capítulo \textsc{xiv},
``Atualidade''. A diferença aqui é de outra ordem. Ocorre que na
primeira edição há um parágrafo final que foi simplesmente suprimido da
segunda. E o que ele diz revela, de fato, a visão do crítico em 1929,
ainda impregnado daquele ar por assim dizer parnasiano, partidário, ao
menos no campo musical, da ``arte pela arte''. Assim dizem as frases do
breve parágrafo:

\begin{quote}
Como é difícil explicar {[}o argumento do valor de todas as formas
musicais do novo século{]}\ldots{} Na verdade eu não pretendo ter descoberto
a pólvora e se que qualquer mal-intencionado pode me contradizer falando
que toda música é tempo, etc. Mas também é bobagem a gente pretender
explicar pra mal-intencionados\ldots{} Sejamos desinteressados, isto é,
sejamos artistas!\ldots{}
\end{quote}

Um ponto a favor da canção, talvez, se encontra nessa supressão. Placar
geral: um honroso \textit{três a um a favor da canção}. Como vimos antes, no momento
inicial de seu percurso como estudioso de música Mário subscrevia a
igualdade entre arte e desinteresse, e isso, em música, só existia no
campo erudito, em sua opinião; essa visada empurrava o resto da criação
musical, aquela do campo popular (e do que ele depois chamaria de
popularesco, quer dizer, a canção popular \textit{praceana}), para fora do
campo da arte, uma vez que se tratava de coisa interessada, isto é, com
interesses mundanos, como certamente era o caso do ``maxixe impresso'' e
do ``samba impresso'', que se vendiam no mercado.

Mas em 1944 Mário estava longe de pensar assim. Como sabemos disso? Não só
pela supressão deste parágrafo na segunda edição do estudo citado, mas
por várias outras manifestações. Uma delas, eloquente ainda que cifrada
e de expressão tortuosa, se encontra no livro \textit{O banquete},
originalmente uma série de textos, publicada em 1944, na coluna ``Mundo
musical'', da \textit{Folha da manhã}, de São Paulo. O nome evidentemente
remete a Platão, uma sugestiva aproximação que fica aqui apenas
mencionada. O livro merece meditação longa, que não será feita aqui --- o
que cabe no momento é alguma menção à figura de Janjão, o compositor,
que interage com outros personagens-tipo, claramente concebidos para
servir como modelos de pensamento.

No capítulo \textsc{ii}, Janjão conversa com Sarah Light, a milionária, e com o
Pastor Fido (``quintanista de direito e vendedor de seguros''). Janjão
defende sua visão da arte musical, e por aqui podemos avaliar algo da
posição de Mário acerca da posição do artista na conjuntura de 1944. ``O
artista que não se preocupa de fazer arte nova é um conformista, tende a
se academizar. (\ldots{}) O artista que não se coloca o problema do fazer
melhor como base da criação é um conformista. Pior! é um folclórico,
como qualquer homem do povo'' --- diz Janjão.

Ao que o Pastor Fido reage perguntando se o artista não está desprezando
o povo e o folclore --- o tema de nosso interesse aqui está vivo, como um
fantasma ou um cadáver insepulto, na conversa de Janjão.

Fido acusa então o compositor de não fazer arte para o povo. Janjão
admite que é isso mesmo, e explica: ``Pelo menos enquanto o povo for
folclórico por definição, isto é: analfabeto e conservador, só existira
uma arte para o povo, a do folclore. {[}\ldots{}{]} O destino do artista erudito
não é fazer arte pro povo, mas pra melhorar a vida.'' E por aí segue, em
conversa que mostra certa dilaceração da sensibilidade de Janjão, que
expõe suas fragilidades como compositor erudito que não aceita as
facilidades da arte dita proletária, nem se rende à arte folclórica,
vivendo entre o apreço pelo internacionalismo abstrato e a perseguição
de uma música nacional, e finalmente nem sabe exatamente quais
alternativas ainda existem.

Aqui pelo menos duas pororocas, talvez três, se encontram e se
potencializam. Num primeiro nível, Janjão sabe que se trata de uma briga
entre o modernismo e o academicismo (e sabemos qual o lado de Mário
aqui). Num segundo plano temos a luta entre o nacional e o
internacional, já mencionada acima. E num terceiro temos a tensão social
entre o povo e a burguesia (termos que Janjão menciona nesta altura do
ensaio) --- e nesta, especificamente, o posicionamento é menos fácil,
porque o povo ``é a fonte'' para a criação, mas é folclórico, ou
``ainda'' folclórico, e de outra parte a burguesia, ou as classes altas
e cultas em geral, é quem tem o gosto apurado e saberá apreciar a arte
dos Janjões.

O assunto, o âmbito de argumentação de Janjão é o da música erudita, não
o da canção popular que aqui estamos considerando. Para esse compositor,
que é Janjão e é Mário, o ideal é que a arte musical brasileira seja
nacional, ``um nacional que difere o folclore, mas que o transubstancia,
porque se trata de música erudita.'' O ponto estaria em encontrar uma
``melódica brasileira, uma polifonia brasileira'', mas sabendo que o
ponto realmente distintivo seria a ``síncopa'', em busca de ``uma
rítmica brasileira em que as síncopas fossem uma constância do
movimento.''

Que lugar teria, nesse quadro conceitual, a canção popular, aquela que
não é folclórica nem erudita, que vende bem, que é impressa e gravada?
Aquela que repete standards e portanto parecia, a um estudioso como
Mário, não merecer a permanência na memória da cultura brasileira, mas
que, dizemos nós, teima em mante-se relevante, formando mesmo um amplo
conjunto de peças breves e significativas para a cultura do país?

É certo que para Mário essa canção popular podia até existir, mas não
era relevante culturalmente, segundo os critérios mais fortes em seu
ideário. É o que se vê nos derradeiros textos do crítico, na mesma
coluna ``Mundo musical''. Num texto de janeiro de 1944, ele saúda o fato
de o Brasil ter entrado em guerra contra o nazismo, e o crítico pensa em
dar um balanço sobre o esforço de guerra por parte da música brasileira.
Nesta altura tão nobre do raciocínio, meio que do nada, o leitor tropeça
nessa joia:

\begin{quote}
Eu até não sei por que me lembrei agora daquele grito agônico e tão
sublime do samba: ``Ai, meu Deus, que saudade da Amélia! Aquilo sim é
que era mulher!\ldots{}'' Porém esta citação também serve para nos conduzir
ao problema mais virulento do esforço musical de guerra: a contribuição
dos compositores.
\end{quote}

Sem querer psicanalisar o escritor, é preciso ver que a evocação do
samba de Ataúlfo Alves e Mário Lago, lançado pouco tempo antes, parece
ser uma daquelas intrusões realmente inesperadas, quase como num sonho.
Porque nem o samba tem qualquer coisa a ver com o dito esforço de
guerra, nem a figura dessa Amélia tem qualquer relação com engajamento
ou algo assemelhado. Talvez a única ponte, o único laço da lembrança do
samba com o tema esteja na palavra \textit{saudade} e, como diz Mário, na
espécie de \textit{grito agônico} da voz que canta a saudade da Amélia ---
como se Mário associasse esse grito sambado com seu grito mudo de
saudade de um tempo anterior à guerra, anterior ao horror do nazismo.

Mas o que importa aqui não é nada disso, e sim o fato de que o grito do
autor se expressou numa canção popular --- aquela que não é mais folclore
nem chega ao mundo erudito, que vende e parece perecível, que enfim não
tem muito valor.

Quer dizer: não tem, mas tem. Na hora do aperto, foi o samba que salvou
o crítico, foi o samba que deu voz a um sentimento forte, relevante,
inadiável.

Aqui o Mário engajado briga com o Mário já modernista mas que pensava,
anos antes, que a arte boa e válida era aquela desinteressada. ``O
compositor `puro' é um errado e um pernicioso que devia ser expulso da
República'' --- diz este Mário alinhado com Platão e com Lênin.

Em abril do mesmo 1944, o crítico volta seu canhão contra o padre José
Maurício, que nada tinha que ver com guerra ou coisa semelhante.
Reconhecendo primeiro que José Maurício tinha sido ``o maior compositor
de música religiosa que o Brasil já possuiu'', evoca depois suas
modinhas, que valem\ldots{} nada! O juízo sumário negativo se deve ao fato de
que ``embora mulato da maior mulataria, escuro e pixaim, ele nada
representa, ou pouco, o valor \textit{negro forro} de nossas idiossincrasias
raciais'', ou, mais amplamente, ele ``não representa os problemas do
Brasil, senão como colono'' --- vale sublinhar que \textit{colono}, aqui,
significa o que hoje chamamos \textit{colonizador}. O padre era, então, o
``anti-Inconfidente típico''.

Isso quer dizer, trocado em miúdos, que José Maurício não pode ser posto
na conta (modernista \textit{marioandradina}, claro) de um compositor válido
culturalmente. Chama a atenção o grande anacronismo do juízo, para nem
falar de outras dimensões. É como se José Maurício, sendo mulato,
devesse ter tido uma consciência composicional que definitivamente não
estava disponível em seu tempo de vida (ele viveu entre 1767 e 1830!).
Para conferir, basta comparar sua obra com a de gente de sua mesma
geração, perguntando o que havia de representação válida na obra dessa
gente, como o (irrelevante) poeta José Bonifácio (1763--1838). Aliás,
pensar nele como ``anti-Inconfidente'' já é um anacronismo --- a geração
que liderou aquele frustrado movimento havia nascido vinte ou mais anos
antes dele.

Toda essa ambígua repulsão pelo campo da canção (ou da música, mais
amplamente) popular impressa e vendável, como as modinhas do padre ou o
samba da Amélia, conflui para o texto final que Mário publica na coluna
``Mundo musical'' no dia 8 de fevereiro de 1945, a derradeira
contribuição dele para o jornal --- morreria no dia 25 do mesmo mês. O
tópico selecionado para o texto vem na primeira linha: ``Popular e
popularesco''. Estamos portanto mergulhados no mundo da canção popular,
mais uma vez.

\begin{quote}
Uma diferença que, pelo menos em música, ajuda bem a distinguir o que é
apenas popularesco, como o samba carioca, do que é verdadeiramente
popular, verdadeiramente folclórico, como o ``Tutu Marambá'', é que o
popularesco tem por sua própria natureza a condição de se sujeitar à
moda. Ao passo que na coisa folclórica, que tem por sua natureza ser
\textit{tradicional} (mesmo transitoriamente tradicional), o elemento moda, a
noção de moda está excluída.
\end{quote}

O esforço de Mário é evidente --- e é inócuo. Ali onde parecia que ia-se
encontrar um critério sólido e nítido, a distinção entre a moda,
transitória, e a cultura tradicional, o que de fato encontramos é mais
hesitação, como se lê nas aspas para a palavra \textit{tradicional} e no
parêntese que vem logo a seguir, com o oxímoro \textit{transitoriamente
tradicional}. O analista e cientista do folclore é menos analista do
que juiz, como se pode constatar. A confusão é vigorosa e
intransponível, dentro dos parâmetros do crítico.

A agonia conceitual aumenta quando Mário tenta equacionar outras
dimensões do fenômeno --- vendo a coisa desde hoje e desde fora dos
marcos conceituais do crítico, seria mais fácil simplesmente admitir que
o que chama de popularesco tem sim valor cultural relevante. Mas não;
veja-se a passagem seguinte:

\begin{quote}
Diante duma marchinha de Carnaval, diante dum \textit{fox-trot} que já
serviram, que já tiveram seu tempo, seu ano, até as pessoas incultas, até
mesmo as pessoas folclóricas da população urbana, reagem, falando que
``isso foi do ano passado'' ou que ``isso é música que passou''. {[}\ldots{}{]}
Ao passo que esse mesmo povo urbano, mesmo sem ser analfabeto, mesmo sem
ser folclórico, jamais dirá isso escutando na macumba um canto de Xangô
que conhece de menino, uma melodia de Bumba-meu-boi sabida desde sempre,
e um refrão de coco na praia, que no entanto são festas anuais, como o
Carnaval.
\end{quote}

São muitas dimensões complexas, aqui reduzidas, por extrema compressão,
a um único par opositivo, folclore \textit{versus} moda --- até mesmo se fala de
``pessoas folclóricas'', numa outra compressão, agora sociológica, que
empurra todo um conjunto de indivíduos para uma única e irreversível
posição!

Mas aqui estamos enfrentando outro tipo de problema --- a transitoriedade
versus a perenidade. Para Mário, é claro que o que chama de popularesco
está no primeiro caso, ao passo que o folclore e o erudito estão no
segundo, e essa distinção parece ao crítico suficiente para mais uma vez
denegar valor à canção popular: ``o documento popularesco, pelo seu
semi-eruditismo, implica civilização, implica progresso, e com isso a
transitoriedade, a velhice, a moda.''

Dá o que pensar essa rejeição. O que diria Mário da sobrevivência, por
gerações a fio, de sambas e marchinhas de carnaval, o que diria da força
cultural, até mesmo no campo letrado, de Noel Rosa, de Cartola, de
Caymmi, de tantos outros, que nada têm dessa esquisitice que Mário chama
\textit{semi-eruditismo}, porque simplesmente não corriam a mesma corrida do
erudito, como ele desejava ou imaginava?

\section{a força da canção}

É certo que não podemos deixar de ler Mário de Andrade quando se trata
de pensar sobre os caminhos e destinos da música no Brasil, em qualquer
de suas modalidades. Mas também é certo que o crítico se relacionou mal,
para dizer de maneira branda, com a canção popular que se sintetizou e
se validou (quando menos, mercadologicamente) no período de vida do
estudioso. Para ele, ou bem se tratava do mundo da música de concerto,
em que brilhou Villa-Lobos atendendo aos critérios do crítico, ou bem da
música comunitária, espontânea, folclórica --- sem espaço conceitual
positivo para a canção que aqui nos interessou.

Será o caso de modular a compreensão dessa relação ruim entre ele e a
canção, lembrando as variáveis em jogo. De um lado, ela mesma, a canção,
enquanto síntese formal, estava ainda se estabilizando quando Mário já
tinha se formado em piano e já começava a trabalhar como professor ---
quer dizer, quando já tinha sua rede conceitual formada. E isso no
universo da música, que desde o Barroco, desde Bach, ao menos, já lidava
com uma separação nítida e irrecorrível entre os dois polos, o
escrito/\,erudito e o oral/\,espontâneo. Não admira, em suma, que o jovem
pianista e professor Mário de Andrade agisse como um reacionário diante
de uma forma nova em processo de estabilização, que era a canção
popular.

Acresce que, como mencionado acima, essa canção popular, além de escapar
entre os buracos da rede conceitual marioandradina, tinha como âmbito de
circulação e validação o mercado, dimensão que um proto-sociademocrata
como Mário via com reserva, por mais de um motivo --- talvez, em sua
sensibilidade modernista-paulista militante, essa presença do mercado no
circuito soasse tão nefasta, tão reprovável, quanto era a Academia
Brasileira de Letras para a literatura brasileira.

E com tudo isso, como também ficou assinalado em passagens anteriores, é
claro que Mário se deixou arrastar pela força da canção. Quer maior
atestado da força da canção popular brasileira do que aquele desabafo
cifrado nas saudades da Amélia, na sensibilidade de um crítico musical
que preferia Villa-Lobos ou a cantiga de roda e o cateretê populares?
