\chapter{Ensaio sobre a música brasileira}\footnote{Este capítulo apresenta a transcrição com grafia atualizada da primeira parte do \textit{Ensaio sobre a música brasileira}, de Mário de Andrade, publicado pela primeira vez em 1962 no volume \textsc{vi} das Obras Completas de Mário de Andrade pela Livraria Martins. No livro, existe uma segunda parte que conta com partituras e letras de várias modalidades de canções populares brasileiras: canto infantil, cantos de trabalho, danças etc. E também o que Mário chama de \textit{música individual}, por vezes colhida diretamente com informantes ou com pesquisadores colaboradores.}

\chapter{«Gravação nacional»}\footnote{Foi transcrita, com ortografia atualizada, a crônica
``Gravação Nacional'', publicada no Diário Nacional em 10 de maio de 1930. Conferir \textit{A música popular brasileira na vitrola de Mario de Andrade}. São Paulo: Sesc, 2004.}

\chapter{«Carnaval tá aí»}\footnote{Esta crônica foi escrita em 18 de janeiro de 1931 para o Diário Nacional, transcrita aqui com ortografia atualizada. Publicada no livro \textit{A música popular brasileira na vitrola de Mario de Andrade}.}}

\chapter{A música e a canção populares no Brasil}\footnote{Capítulo do livro \textit{Ensaio sobre a música brasileira}, transcrito aqui com ortografia atualizada.}

\chapter{A pronúncia cantada e o problema do nasal brasileiro através dos discos}\footnote{Este trabalho foi apresentado por Mário de Andrade no 1º Congresso da Língua Nacional Cantada, de 1937, e discute dicção entre cantores e cantoras do \textit{bel canto}. Foi depois publicado como um capítulo do \textit{Aspectos da música brasileira}, volume \textsc{xi} de suas Obras Completas. Cf. \textit{Aspectos da Música Brasileira}, organização de 2012 da Nova Fronteira.}

\chapter{Evolução social da música no Brasil}\footnote{Capítulo do livro \textit{Aspectos da música brasileira}, transcrito aqui com ortografia atualizada.}

\chapter{Romantismo musical}\footnote{Capítulo da obra \textit{O baile das quatro artes}, publicada em 1943. O livro reúne sete textos, entre palestras e artigos escritos no período de 1932 até 1942. Conferir \textit{O baile das quatro artes}, Editora Garnier, 2005.}

\chapter{Trecho de carta a Moacir Werneck de Castro}\footnote{Conferir \textit{A música popular brasileira na vitrola de Mario de Andrade}.}

\chapter{Música Popular Brasileira}\footnote{Capítulo presente no livro \textit{Pequena história da música}, que aborda, além da música popular brasileira, música erudita, instrumental, clacissismo e romantismo e música da Antiguidade, apenas para citar alguns temas. Conferir \textit{Pequena história da música}, publicado em 2015 pela Nova Fronteira.}

\chapter{Dicionário Musical Brasileiro\footnote{Mário de Andrade compilou neste dicionário os principais termos da música brasileira, com todos os vocábulos próprios e características históricas curiosas. Conferir \textit{Dicionário musical brasileiro}, Ministério da Cultura, 1989.}
