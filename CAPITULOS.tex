\chapterspecial{Ensaio sobre a música brasileira}{1928\footnote{Este capítulo apresenta a transcrição com grafia atualizada da primeira parte do \textit{Ensaio sobre a música brasileira}, de Mário de Andrade, publicado pela primeira vez em 1962 no volume \textsc{vi} das Obras Completas de Mário de Andrade pela Livraria Martins. No livro, existe uma segunda parte que conta com partituras e letras de várias modalidades de canções populares brasileiras: canto infantil, cantos de trabalho, danças etc. E também o que Mário chama de \textit{música individual}, por vezes colhida diretamente com informantes ou com pesquisadores colaboradores.}}

\chapter{«Gravação nacional»}\footnote{Foi transcrita, com ortografia atualizada, a crônica
``Gravação Nacional'', publicada no Diário Nacional em 10 de maio de 1930. Cf. \textit{A música popular brasileira na vitrola de Mario de Andrade}. São Paulo: Sesc, 2004.}}{}

\chapter{«Carnaval tá aí»}\footnote{Esta crônica foi escrita em 18 de janeiro de 1931, para o Diário Nacional, transcrita aqui com ortografia atualizada. Publicada no livro \textit{A música popular brasileira na vitrola de Mario de Andrade}.}}

\chapter{A música e a canção populares no Brasil}\footnote{Transcrito, com ortografia atualizada, o capítulo do livro \textit{Ensaio sobre a música brasileira}. O livro conta também com partituras de várias modalidades, letras de diversas canções, incluindo cantos infantis e danças. Cf. ``Ensaio sobre a música brasileira''. \textit{Obras Completas de Mário de Andrade}, vol.\,\textsc{vi}. São Paulo: Livraria Martins, 1962.}}

\chapter{A pronúncia cantada e o problema do nasal brasileiro através dos discos}\footnote{Capítulo do livro \textit{Aspectos da música brasileira} (vol.\,\textsc{xi} das Obras Completas). O trabalho foi apresentado no Primeiro Congresso da Língua Nacional Cantada, de 1937, e discute dicção entre cantores e cantoras do belcanto. Cf. \textit{A Pronúncia cantada e o problema do nasal brasileiro através dos discos}. \textit{Aspectos da Música Brasileira}. Rio de Janeiro: Nova Fronteira (2012).}}{}

\chapterspecial{Evolução social da música no Brasil}\footnote{Transcrevemos aqui, com ortografia atualizada, o ensaio Evolução Social da Música no Brasil que está presente no livro \textit{Aspectos da música brasileira}. Cf. \textit{Evolução social da música no Brasil}. \textit{Aspectos da Música Brasileira}. Rio de Janeiro: Nova Fronteira (2012).}}{}

\chapterspecial{Romantismo Musical}{1941\footnote{\textit{Romantismo Musical}. O baile das quatro artes, 1º edição, Editora Garnier, 2005. Transcrevemos, com ortografia atualizada, o capítulo intitulado Romantismo Musical da obra O baile das quatro artes, publicada em 1943. Este livro reúne textos que não foram originalmente pensados para o livro. Assim, são sete palestras e artigos escritos num período que vai desde 1932 até 1942.}}{}

\chapterspecial{Trecho de carta a Moacir Werneck de Castro}{1942\footnote{\textit{Trecho de carta a Moacir Werneck de Castro}. A música popular brasileira na vitrola de Mario de Andrade, 1º edição, São Paulo: Sesc, 2004.}}{}

\chapterspecial{Música Popular Brasileira}{1944\footnote{Música Popular Brasileira. Pequena História da Música, Rio de Janeiro: Nova Fronteira, 2015. Transcrevemos o capítulo presente no livro Pequena História da Música, uma obra inteiramente dedicada à análise da música realizada por Mario de Andrade com sua perspicaz sensibilidade e dedicada pesquisa. O livro aborda além da Música Popular Brasileira, a também música erudita, a música instrumental, o clacissimos e romantismo, e a música da Antiguidade, só para citar alguns temas.}}

\chapter{Dicionário Musical Brasileiro\footnote{\textit{Dicionário musical brasileiro}. Mario de Andrade. Ministério da Cultura, 1989. O escritor, pesquisador e músico Mário de Andrade compilou neste dicionário os principais termos da música brasileira com todos os seus vocábulos próprios e características históricas curiosas.}}